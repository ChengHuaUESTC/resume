% !TEX TS-program = xelatex
% !TEX encoding = UTF-8 Unicode
% !Mode:: "TeX:UTF-8"

\documentclass{resume}
\usepackage{zh_CN-Adobefonts_external} % Simplified Chinese Support using external fonts (./fonts/zh_CN-Adobe/)
%\usepackage{zh_CN-Adobefonts_internal} % Simplified Chinese Support using system fonts
\usepackage{linespacing_fix} % disable extra space before next section
\usepackage{cite}

\begin{document}
\pagenumbering{gobble} % suppress displaying page number

\name{程华}

\basicInfo{
  \email{chenghua.root@gmail.com} \textperiodcentered\
  \phone{(+86) 158-822-04448} \textperiodcentered\
}
\section{\faGraduationCap\  教育背景}
\datedsubsection{\textbf{电子科技大学}}{2012.9 -- 2015.6}
\textit{硕士研究生}\ 计算机系统结构
\datedsubsection{\textbf{电子科技大学}}{2007.9 -- 2011.6}
\textit{学士}\ 软件工程

\section{\faBuilding\  工作经历}
\datedsubsection{\textbf{美团网}}{2015.7 -- 至今}
\textit{}\ 后台开发工程师/分布式系统开发工程师

\section{\faUsers\  项目经历}
\datedsubsection{\textbf{美团}  北京}{2015年9月 -- 至今}
\role{美团云图片处理系统}{}
美团云图片处理系统是构建于美团云对象存储的图片处理系统。支持缩放、裁剪、旋转、图文水印、格式转换、拼图和样式管理等功能;功能全面,使用方便,已被公司内外广泛使用。整个项目采用Go语言+ImageMagick库开发完成。
\begin{itemize}
  \item 独立负责整个系统的设计、开发和维护。除功能开发外,还包括:
  \item 多进程改造:受限于ImageMagick库对单进程的并发处理不够友好,后面改为单监听端口+多进程+多routine模型,系统并发能力提升10倍。
  \item 开发了对比测试模块,支持跟阿里云的图片服务做对比测试。
  \item 集群为双机房部署,每个机房为独立缓存系统,提升系统的可靠性。
  \item API地址:https://www.mtyun.com/doc/api/mss/mss/tu-pian-chu-li-fu-wu-api
\end{itemize}

\datedsubsection{\textbf{美团}  北京}{2015年10月 -- 至今}
\role{美团云新一代对象存储系统}{}
MSS(美团云对象存储服务)是经过美团内部反复验证的,高可靠、高可用、海量、安全的对象存储服务。兼容S3协议,允许指定对象的持久化级别(多副本或者纠删码);支持光纤互联机房间的异地备份和本地读写优化;保证副本之间强一致;保证在集群掉电或少于设计数量的磁盘损坏的情况下数据不丢失;自动化的扩容与故障隔离;高并发情况下毫秒级别的写入性能以及较高的网络和磁盘带宽;集群高可用性。
\begin{itemize}
  \item 了解并参与系统的整体架构,参与设计和实现proxy(协议模块),整体响应和处理流程,包括请求信息处理、元数据导入封装和请求路由。
  \item 结合元数据缓存管理、设计并发读写、优先本地读取等以降低读取延迟和提高读写响应带宽,利用分布式限速降低系统不可用的风险。
  \item 使用pprof/go-torch工具对proxy模块进行问题定位和性能调优。
  \item 优化store模块的核心读写逻辑:采用线程队列、批量提交等技术提高写入性能,使用强一致的副本策略和冻结技术保证数据的一致性。
  \item 设计EC在线修复流程,结合机房(减少跨机房)、计算修复节点(对修复内容缓存)等降低在线修复延迟,结合心跳信息通知管理节点开启离线修复以恢复EC副本。
  \item 编写和维护项目多机自动化测试、单机集成测试等脚本程序。
\end{itemize}

% Reference Test
%\datedsubsection{\textbf{Paper Title\cite{zaharia2012resilient}}}{May. 2015}
%An xxx optimized for xxx\cite{verma2015large}
%\begin{itemize}
%  \item main contribution
%\end{itemize}

\section{\faCogs\  技能}
% increase linespacing [parsep=0.5ex]
\begin{itemize}[parsep=0.5ex]
  \item 使用Go、C编写程序。了解Go的原理,有Go语言项目开发经验。
  \item 多年linux服务端开发经验,熟悉linux环境编程和系统多线程编程,熟悉网络协议(HTTP、TCP/IP)和网络编程。
  \item 了解分布式系统的基本原理,主要侧重分布式存储系统,有一定的论文和项目积累,了解raft算法,熟悉常见的分布式存储系统的设计。
  \item 熟练使用Git进行代码管理,可以处理常见的Git问题。
  \item 熟悉linux系统的使用,掌握常用的操作命令和运维命令,可以使用shell脚本和python开发一些运维工具。
  \item 关注分布式数据库的最新技术,喜欢看一些分布式数据库相关的文档资料。
  \item 追求技术,了解一些开源系统的的实现原理。
\end{itemize}

\section{\faInfo\  其他}
% increase linespacing [parsep=0.5ex]
\begin{itemize}[parsep=0.5ex]
  \item 性格:热爱参与技术分享和交流,积极乐观,能进行良好的沟通合作。
  \item 运动: 爱好运动,蛙泳/自由泳,参与公司足球比赛,读书期间曾获“成电杯”研究生足球比赛亚军。
  \item 语言: 英语 - 熟练(六级)
\end{itemize}

%% Reference
%\newpage
%\bibliographystyle{IEEETran}
%\bibliography{mycite}
\end{document}
