% !TEX TS-program = xelatex
% !TEX encoding = UTF-8 Unicode
% !Mode:: "TeX:UTF-8"

\documentclass{resume}
\usepackage{zh_CN-Adobefonts_external} % Simplified Chinese Support using external fonts (./fonts/zh_CN-Adobe/)
%\usepackage{zh_CN-Adobefonts_internal} % Simplified Chinese Support using system fonts
\usepackage{linespacing_fix} % disable extra space before next section
\usepackage{cite}

\begin{document}
\pagenumbering{gobble} % suppress displaying page number

\name{林萌}

\basicInfo{
  \email{lemon9010@gmail.com} \textperiodcentered\
  \phone{(+86) 135-529-87847} \textperiodcentered\
}
\section{\faGraduationCap\  教育背景}
\datedsubsection{\textbf{电子科技大学}}{2013 -- 2016}
\textit{硕士研究生}\ 计算机应用技术,成绩:前50/200
\datedsubsection{\textbf{电子科技大学}}{2009 -- 2013}
\textit{学士}\ 信息安全,成绩:top3\%

\section{\faUsers\ 项目经历}
\datedsubsection{\textbf{美团} 北京}{2016年3月 -- 至今}
\role{全职}{分布式存储研发工程师}
MSS(美团云对象存储服务)是经过美团内部反复验证的,高可靠、高可用、海量、安全的对象存储服务。兼容S3协议,允许指定对象的持久化级别(多副本或者纠删码);支持光纤互联机房间的异地备份和本地读写优化;保证副本之间强一致;保证在集群掉电或少于设计数量的磁盘损坏的情况下数据不丢失;自动化的扩容与故障隔离;高并发情况下毫秒级别的写入性能以及较高的网络和磁盘带宽;集群高可用性。
\begin{itemize}
  \item 了解系统的整体架构,参与设计和实现store(存储节点)黑名单机制(处理节点间网络故障和抖动),store异步数据检测校验机制和gc流程。
  \item 开发store模块的核心读写逻辑:采用线程队列、批量提交等技术提高写入性能,使用强一致的副本策略和冻结技术保证数据的一致性。
  \item 开发store模块的元数据管理:主要包括副本(固定大小256M)的创建、多副本管理、副本状态的维护等工作,保证多线程写入的数据安全。
  \item 对store模块进行一些性能优化:使用火焰图、日志分析、tracelog等手段对性能问题进行定位。
  \item 元数据存储使用分布式数据库系统,编写go数据库客户端相关的业务逻辑,保证元数据更新和查询的正确性。
  \item 解决go客户端高并发情况下的一些死锁问题,使用相关pprof工具对元数据客户端进行问题定位和性能调优。
  \item 编写元数据(存储在分布式数据库中)冷备份模块:主要利用元数据特性进行高效select扫表,使用go的高并发实现数据快速备份。
  \item 编写和维护项目的一些编译安装脚本、多机自动化测试脚本、单机测试脚本等脚本程序,对原有脚本进行一些改进。
\end{itemize}

\datedsubsection{\textbf{分布式系统实验室} 电子科技大学}{2014年10月 -- 2015年11月}
\role{主研}{分布式列式内存数据库存储引擎}
分布式列式内存数据库主要面向OLAP(联机分析处理)的应用,用来高效存储和分析海量数据。该系统实现从外部数据源(如oracle系统)快速并行导入数据到本系统中,在内存中存储导入数据,并能够对外提供数据快速查询服务。
\begin{itemize}
  \item 调研常见的分布式存储系统和相关理论技术,参与系统的总体设计,包括系统整体架构,交互协议和流程,数据分布,扩展性,容错和故障恢复等。
  \item 设计和实现总控节点(master)的功能。主要任务包括:(1)实现和导入集群的交互,调度导入任务,监控任务执行和控制任务重新调度。(2)控制数据分布和数据迁移,控制存储节点的内存负载,监控存储节点状态。(3)保存元数据信息,包括索引信息,副本相关信息,访问权限信息等。(4)使用raft算法实现master节点的高可用。
  \item 设计实现数据存储节点的列数据的内存索引结构。该索引结构实现了字典到行号、行号到字典的快速映射。采用位压缩技术实现类似于vector的存储容器,高效利用内存。设计数据追加算法,实现数据的快速追加。
\end{itemize}

% Reference Test
%\datedsubsection{\textbf{Paper Title\cite{zaharia2012resilient}}}{May. 2015}
%An xxx optimized for xxx\cite{verma2015large}
%\begin{itemize}
%  \item main contribution
%\end{itemize}

\section{\faCogs\ IT 技能}
% increase linespacing [parsep=0.5ex]
\begin{itemize}[parsep=0.5ex]
  \item 多年C/C++编程经验,良好的编程规范,熟悉gcc,g++,automake等编译链接工具,熟练使用gdb进行程序调试,有大型C/C++项目开发经验。
  \item 熟练使用go编写程序,熟悉go的一些原理,有go语言项目开发经验、问题诊断和调优经验。
  \item 多年linux服务端开发经验,熟悉linux环境编程和系统多线程编程。
  \item 熟悉linux系统的使用,掌握常用的操作命令和运维命令,可以使用shell脚本和python开发一些运维工具。
  \item 了解分布式系统的基本原理,主要侧重分布式存储系统,有一定的论文和项目积累,熟悉raft算法和paxos算法,熟悉常见的分布式存储系统的设计。
  \item 熟练使用git进行代码管理,可以处理常见的git问题。
  \item 了解MySQL协议,熟悉常见的数据库操作。
  \item 热爱开源,追求技术,了解一些开源系统的的实现原理,阅读一些开源系统代码。
\end{itemize}

\section{\faHeartO\ 获奖情况}
\datedline{电子科技大学ACM竞赛三等奖}{2011 年6 月}
\datedline{三星应用开发助跑计划}{2013 年3 月}

\section{\faInfo\ 其他}
% increase linespacing [parsep=0.5ex]
\begin{itemize}[parsep=0.5ex]
  \item 技术博客: http://lemon0910.github.io
  \item GitHub: https://github.com/lemon0910
  \item 性格:热爱开源,参与技术分享和交流,积极乐观,能与人进行良好的合作和沟通。
  \item 语言: 英语 - 熟练(六级)
\end{itemize}

%% Reference
%\newpage
%\bibliographystyle{IEEETran}
%\bibliography{mycite}
\end{document}
